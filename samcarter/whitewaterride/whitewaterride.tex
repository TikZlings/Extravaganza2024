\documentclass[aspectratio=169]{beamer}

\usepackage{tikzducks}

\setbeamertemplate{navigation symbols}{}
\setbeamertemplate{background canvas}{\includegraphics[width=\paperwidth]{water-\thepage.jpg}}
\graphicspath{{include/}}

% trick taken from https://topanswers.xyz/tex?q=1989
\tikzset{
    use page relative coordinates/.style={
        shift={(current page.south west)},
        x={(current page.south east)},
        y={(current page.north west)}
    },
}

\newcommand{\floatingduck}[2]{
    \duck[
      scale=1.5,
      xscale=-1,
      yshift=#2-\insertoverlaynumber*0.04cm,
      xshift=#1-\insertoverlaynumber*0.2cm
    ] 
}


\begin{document}

\begin{frame}
  \begin{tikzpicture}[remember picture, overlay]
  
    \floatingduck{4cm}{0cm}
    \floatingduck{14cm}{1cm}
    \floatingduck{20cm}{3cm}
    \floatingduck{25cm}{5cm}
    \floatingduck{35cm}{8cm}
    \floatingduck{40cm}{9cm}
    \floatingduck{50cm}{11cm}    
    \floatingduck{70cm}{15cm}
    \floatingduck{73cm}{15cm}
    \floatingduck{90cm}{18cm}  
    \floatingduck{100cm}{19cm}     
    \floatingduck{107cm}{19.5cm}                   
    
    % credit for background video
    \node[white,text width=.9\paperwidth,font=\tiny,align=center] at ([yshift=0.35cm]current page.south) {Background video by @Hucklebarry (\url{https://pixabay.com/videos/water-river-current-nature-outdoor-32558/})};    

  \end{tikzpicture}
  \pause[632]
\end{frame}


\end{document}