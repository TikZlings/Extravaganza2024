% !Mode:: "TeX:UTF-8:Main"

% slow play!
% music:
% 0:00 - 0:30
% https://www.youtube.com/watch?v=WYeDsa4Tw0c&t=5s
 
\documentclass[aspectratio=169]{beamer}
\usepackage[svgnames,x11names]{xcolor}
\usepackage{tikz}
\usetikzlibrary{ducks}
\usetikzlibrary{tikzlings}
\usetikzlibrary {decorations.markings}

\setbeamertemplate{navigation symbols}{}
\setbeamertemplate{background canvas}{}

\ExplSyntaxOn
\AddToHook{shipout/background}%
 {
   \put(0,-\paperheight){%
      \includegraphics[page=\int_eval:n{\int_mod:nn{\value{page}}{10}+1}]{starsky}}
   \put(0,-\paperheight+1cm){\scalebox{0.3}{\usebox\mybox}}
   \put(1cm,-\paperheight+2.5cm){\scalebox{0.4}{\usebox\mybox}} 
   \put(0.1cm,-\paperheight+3cm){\scalebox{0.25}{\usebox\mybox}}    
   \put(1cm,-\paperheight+2cm){\scalebox{0.3}{\usebox\mybox}}
   \put(3cm,-\paperheight+1.5cm){\scalebox{0.4}{\usebox\mybox}} 
   \put(1cm,-1.8cm){\scalebox{0.25}{\usebox\mybox}} 
   \put(2cm,-1.9cm){\scalebox{0.26}{\usebox\mybox}} 
   \put(0cm,-1.9cm){\scalebox{0.29}{\usebox\mybox}} 
 }
\AddToHook{shipout/foreground}%
 {
   \put(\paperwidth-3cm,-\paperheight+0.5cm){\scalebox{0.35}{\usebox\mybox}}   
   \put(\paperwidth-5cm,-\paperheight+0.2cm){\scalebox{0.3}{\usebox\mybox}} 
   \put(\paperwidth-6.5cm,-\paperheight+0.3cm){\scalebox{0.25}{\usebox\mybox}}
   \put(\paperwidth-8cm,-\paperheight+0.1cm){\scalebox{0.33}{\usebox\mybox}} 
   \put(\paperwidth-10cm,-\paperheight){\scalebox{0.25}{\usebox\mybox}}   
   \put(\paperwidth-12cm,-\paperheight+0.1){\scalebox{0.28}{\usebox\mybox}}
   \put(\paperwidth-3cm,-3cm){\scalebox{0.28}{\usebox\mybox}}
   \put(\paperwidth-2cm,-3cm){\scalebox{0.28}{\usebox\mybox}}
   \put(\paperwidth-3.5cm,-3.4cm){\scalebox{0.27}{\usebox\mybox}}
 }   
\ExplSyntaxOff

\begin{document}
\newsavebox\mybox
\savebox\mybox
 {\tikz{\filldraw[DarkGoldenrod4!50!black] (-.5,-.5) rectangle ++(1,1.5);
	\shade[top color=white,bottom color=Green4!50!black] (0,1)
	to[bend right] ++(4,0)
	to[bend left] ++(-3,2)
	to[bend right] ++(2,0)
	to[bend left] ++(-2.5,2)
	to[bend right] ++(1,0)
	to[bend left] ++(-1.5,2)
	-- cycle;
	\shade[top color=white,bottom color=Green4!50!black] (0,1)
	to[bend left] ++(-4,0)
	to[bend right] ++(3,2)
	to[bend left] ++(-2,0)
	to[bend right] ++(2.5,2)
	to[bend left] ++(-1,0)
	to[bend right] ++(1.5,2)
	-- cycle;
	\draw[white,thick] (0,1)
	to[bend right] ++(4,0)
	to[bend left] ++(-3,2)
	to[bend right] ++(2,0)
	to[bend left] ++(-2.5,2)
	to[bend right] ++(1,0)
	to[bend left] ++(-1.5,2);
	%-- cycle;
  	\draw[white,thick] (0,1)
	to[bend left] ++(-4,0)
	to[bend right] ++(3,2)
	to[bend left] ++(-2,0)
	to[bend right] ++(2.5,2)
	to[bend left] ++(-1,0)
	to[bend right] ++(1.5,2);
}}	
\newcounter{stepcnt}
%\end{document}
\makeatletter
\ExplSyntaxOn
\seq_new:N\g_my_tikzlings_seq
\seq_gset_from_clist:NN\g_my_tikzlings_seq\tikzlings@names@clist
\seq_gput_left:Nn\g_my_tikzlings_seq{duck}
\newcommand\processtikzlings
 {\seq_map_indexed_inline:Nn\g_my_tikzlings_seq
   {
   \edef\loopvariable{\inteval{\the\value{page}-##1*20}}
   \ifnum\loopvariable>-10 
  \ifnum\loopvariable <200  
  \path[]  
    (-0.6\textwidth,-0.1\textheight)  -- 
    (\fpeval{\loopvariable/200}\textwidth,
     \fpeval{\loopvariable/200*0.3}\textheight)
   pic[scale=\scalefactor,xscale=-1] {##2};
  \else
  \ifnum\loopvariable <400
   \path[]  
     (\textwidth,0.3\textheight)  -- 
     (\fpeval{1-(\loopvariable-200)/200}\textwidth,
      \fpeval{(0.3+(\loopvariable-200)/200*0.3)}\textheight)
     pic[scale=\scalefactor] {##2};
    \else
     \path[]  
      (0,0.6\textheight)  -- 
      (\fpeval{(\loopvariable-400)/200}\textwidth,
       \fpeval{(0.6+(\loopvariable-400)/200*0.2)}\textheight)
      pic[scale=\scalefactor,xscale=-1] {##2};
   \fi   
  \fi 
  \fi
  } 
 }
\ExplSyntaxOff


\tikzset{
  laser beam action/.style={
    line width=\pgflinewidth+.2pt,draw opacity=.1,draw=#1,
  },
  laser beam recurs/.code 2 args={%
    \pgfmathtruncatemacro{\level}{#1-1}%
    \ifnum\level=0 %
    \tikzset{preaction={laser beam action=#2}}%
    \else
    \tikzset{preaction={laser beam action=#2,laser beam recurs={\level}{#2}}}
    \fi
    }
  ,
  laser beam/.style={preaction={laser beam recurs={10}{#1}},draw opacity=1,draw=#1},
}
\begin{frame}
  \begin{tikzpicture}[%
  remember picture,
  ] 
  \def\scalefactor{\fpeval{1-\loopvariable/1200}} 
  \path[use as bounding box](0,0)rectangle(\textwidth,\textheight-15pt);   
  \path[draw=yellow!80!white!90!black,opacity=0.1,line width=1.3cm,rounded corners]
   (-0.6\textwidth,-0.1\textheight) -- (\textwidth,0.3\textheight)--
   (0,0.6\textheight)--(1.2\textwidth,\textheight);
  \processtikzlings
  %{
%   \edef\loopvariable{\inteval{\the\value{page}-#1*10}}
%   \ifnum\loopvariable>0 
%  \ifnum\loopvariable <200  
%  \path[draw=green, thick]  
%    (-0.6\textwidth,-0.1\textheight)  -- 
%    (\fpeval{\loopvariable/200}\textwidth,
%     \fpeval{\loopvariable/200*0.3}\textheight)
%   pic[scale=\scalefactor] {#2};
%  \else
%  \ifnum\loopvariable <400
%   \path[draw=green, thick]  
%     (\textwidth,0.3\textheight)  -- 
%     (\fpeval{1-(\loopvariable-200)/200}\textwidth,
%      \fpeval{(0.3+(\loopvariable-200)/200*0.3)}\textheight)
%     pic[scale=\scalefactor] {#2};
%    \else
%     \path[draw=green, thick]  
%      (0,0.6\textheight)  -- 
%      (\fpeval{(\loopvariable-400)/200}\textwidth,
%       \fpeval{(0.6+(\loopvariable-400)/200*0.2)}\textheight)
%      pic[scale=\scalefactor] {#2};
%   \fi   
%  \fi 
%  \fi
%  }
%  \edef\loopvariable{\inteval{\the\value{page}-20}}
%  \ifnum\loopvariable>0
%  \ifnum\loopvariable <200  
%  \path[draw=green, thick]  
%    (-0.6\textwidth,-0.1\textheight)  -- 
%    (\fpeval{\loopvariable/200}\textwidth,
%     \fpeval{\loopvariable/200*0.3}\textheight)
%    pic[scale=\scalefactor] {bear};
%  \else
%  \ifnum\loopvariable <400
%   \path[draw=green, thick]  
%     (\textwidth,0.3\textheight)  -- 
%     (\fpeval{1-(\loopvariable-200)/200}\textwidth,
%      \fpeval{(0.3+(\loopvariable-200)/200*0.3)}\textheight)
%     pic[scale=\scalefactor] {bear};
%    \else
%     \path[draw=green, thick]  
%      (0,0.6\textheight)  -- 
%      (\fpeval{(\loopvariable-400)/200}\textwidth,
%       \fpeval{(0.6+(\loopvariable-400)/200*0.2)}\textheight)
%      pic[scale=\scalefactor] {bear};
%   \fi   
%  \fi 
%  \fi     
  \end{tikzpicture}
  %\pause[100]
  \pause[800]
\end{frame}	
	
\end{document}
