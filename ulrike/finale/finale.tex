% !Mode:: "TeX:UTF-8:Main"

% slow play!
% music:
% 0:00 - 0:30
% https://www.youtube.com/watch?v=WYeDsa4Tw0c&t=5s
 
\documentclass[aspectratio=169]{beamer}

\usepackage{tikz}
\usetikzlibrary{ducks}
\usetikzlibrary{tikzlings}
\usetikzlibrary {decorations.markings}

\setbeamertemplate{navigation symbols}{}
\setbeamertemplate{background canvas}{\makebox[\paperwidth]
{%\includegraphics[width=0.9\paperwidth,trim=0cm 0cm 0cm 5cm]
%  {Matterhorn-EastAndNorthside-viewedFromZermatt_landscapeformat-2}
}}

\begin{document}
\newsavebox\mybox
\savebox\mybox{\tikz{\duck}}	
\newcounter{stepcnt}
%\end{document}
\makeatletter
\ExplSyntaxOn
\seq_new:N\g_my_tikzlings_seq
\seq_gset_from_clist:NN\g_my_tikzlings_seq\tikzlings@names@clist
\seq_gput_left:Nn\g_my_tikzlings_seq{duck}
\newcommand\processtikzlings
 {\seq_map_indexed_inline:Nn\g_my_tikzlings_seq
   {
   \edef\loopvariable{\inteval{\the\value{page}-##1*20}}
   \ifnum\loopvariable>0 
  \ifnum\loopvariable <200  
  \path[draw=green, thick]  
    (-0.6\textwidth,-0.1\textheight)  -- 
    (\fpeval{\loopvariable/200}\textwidth,
     \fpeval{\loopvariable/200*0.3}\textheight)
   pic[scale=\scalefactor] {##2};
  \else
  \ifnum\loopvariable <400
   \path[draw=green, thick]  
     (\textwidth,0.3\textheight)  -- 
     (\fpeval{1-(\loopvariable-200)/200}\textwidth,
      \fpeval{(0.3+(\loopvariable-200)/200*0.3)}\textheight)
     pic[scale=\scalefactor] {##2};
    \else
     \path[draw=green, thick]  
      (0,0.6\textheight)  -- 
      (\fpeval{(\loopvariable-400)/200}\textwidth,
       \fpeval{(0.6+(\loopvariable-400)/200*0.2)}\textheight)
      pic[scale=\scalefactor] {##2};
   \fi   
  \fi 
  \fi
  } 
 }
\ExplSyntaxOff
\begin{frame}
  \begin{tikzpicture}[%
  remember picture,
  ] 
  \def\scalefactor{\fpeval{0.5+\loopvariable/1200}} 
  \path[use as bounding box](0,0)rectangle(\textwidth,\textheight-15pt);   
  \path[draw=red]
   (-0.6\textwidth,-0.1\textheight) -- (\textwidth,0.3\textheight)--
   (0,0.6\textheight)--(1.2\textwidth,\textheight);
  \processtikzlings
  %{
%   \edef\loopvariable{\inteval{\the\value{page}-#1*10}}
%   \ifnum\loopvariable>0 
%  \ifnum\loopvariable <200  
%  \path[draw=green, thick]  
%    (-0.6\textwidth,-0.1\textheight)  -- 
%    (\fpeval{\loopvariable/200}\textwidth,
%     \fpeval{\loopvariable/200*0.3}\textheight)
%   pic[scale=\scalefactor] {#2};
%  \else
%  \ifnum\loopvariable <400
%   \path[draw=green, thick]  
%     (\textwidth,0.3\textheight)  -- 
%     (\fpeval{1-(\loopvariable-200)/200}\textwidth,
%      \fpeval{(0.3+(\loopvariable-200)/200*0.3)}\textheight)
%     pic[scale=\scalefactor] {#2};
%    \else
%     \path[draw=green, thick]  
%      (0,0.6\textheight)  -- 
%      (\fpeval{(\loopvariable-400)/200}\textwidth,
%       \fpeval{(0.6+(\loopvariable-400)/200*0.2)}\textheight)
%      pic[scale=\scalefactor] {#2};
%   \fi   
%  \fi 
%  \fi
%  }
%  \edef\loopvariable{\inteval{\the\value{page}-20}}
%  \ifnum\loopvariable>0
%  \ifnum\loopvariable <200  
%  \path[draw=green, thick]  
%    (-0.6\textwidth,-0.1\textheight)  -- 
%    (\fpeval{\loopvariable/200}\textwidth,
%     \fpeval{\loopvariable/200*0.3}\textheight)
%    pic[scale=\scalefactor] {bear};
%  \else
%  \ifnum\loopvariable <400
%   \path[draw=green, thick]  
%     (\textwidth,0.3\textheight)  -- 
%     (\fpeval{1-(\loopvariable-200)/200}\textwidth,
%      \fpeval{(0.3+(\loopvariable-200)/200*0.3)}\textheight)
%     pic[scale=\scalefactor] {bear};
%    \else
%     \path[draw=green, thick]  
%      (0,0.6\textheight)  -- 
%      (\fpeval{(\loopvariable-400)/200}\textwidth,
%       \fpeval{(0.6+(\loopvariable-400)/200*0.2)}\textheight)
%      pic[scale=\scalefactor] {bear};
%   \fi   
%  \fi 
%  \fi     
  \end{tikzpicture}
  %\pause[100]
  \pause[800]
\end{frame}	
	
\end{document}
